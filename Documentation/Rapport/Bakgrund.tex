\mychapter{Bakgrund}

\section{Kryptografi} % Presenting cryptography and some famous algorithms and systems...
Ordet kryptografi härstammar från de två grekiska orden
kryptos som betyder gömd och grafein som betyder skrift.\footfullcite{krypto}
I sin simplaste form handlar kryptografi alltså om att
gömma information. Detta är något som har visat sig på många
olika sätt genom historien från något så simpelt som att skriva
ett medelande i text då många i början inte kunde läsa till
att idag istället använda komplexa algoritmer.\footfullcite{kryptografi-historia-1}
Begreppet kryptografi har dock också fått en utökade betydelse
med tiden då det idag även inkluderar olika metoder för att
säkerställa autenticiteten av informationen och avsändaren.\footfullcite{NE-1}

\subsection{Uppkomst} % Mention things like Caesar cipher
Kryptografins historia kan man nästan säga börjar vid den
tidigaste formen av skrift, vilket grundar sig i de faktum att
de flesta inte kunde läsa. Detta är ju såklart något som förändrats
på senare tid och i takt med de så har även kryptografin utvecklats.
Exempel på utvecklingen går att se så tidigt som 1900 f.Kr då vissa egyptiska
skribenter använde sig utav hieroglyfer på ett avvikande sätt, vilket
troligen då gjordes i syfte att dölja informationen från dom som inte
visste vad det skulle betyda.\footcite{kryptografi-historia-1}

Den tidiga kryptografin är även något som kan observeras hos romarna där
man använde \gls{caesar} och hos grekerna. Där grekernas metod byggde på
att man virade en tejpbit runt någon form av ett cylinderformat objekt
och sedan skrev medellandet på tejpen. När tejpen sedan togs av så är texten % This section might require some modification
oläslig och mottagaren behövde vira upp tejpen på ett cylinderformat objekt
med samma diameter för att läsa det.\footcite{kryptografi-historia-1}

\subsection{Utveckling} % Mention things like DES, RSA and enigma as well as the development of the algorithms and technological advances

\section{AES Uppkomst} % The rise of the AES standard and the Rijndael algorithm

