\mychapter{Bakgrund}

\section{Kryptografi} % Ask for opinion on this section
Ordet kryptografi härstammar från de två grekiska orden
kryptos som betyder gömd och grafein som betyder skrift.\footfullcite{krypto}
I sin simplaste form handlar kryptografi alltså om att
gömma information. Detta är något som har visat sig på många
olika sätt genom historien från något så simpelt som att skriva
ett medelande i text då många i början inte kunde läsa till
att idag istället använda komplexa algoritmer så som \acrshort{aes} \& \acrshort{des}.\footfullcite{kryptografi-historia-1}
Begreppet kryptografi har dock också fått en utökade betydelse
med tiden då det idag även inkluderar olika metoder för att
säkerställa autenticiteten av informationen och avsändaren.\footfullcite{NE-1}

\section{Varför behövs kryptering?} % Why do we need encryption?
\label{sec:varfor-behovs-kryptering}
I takt med utvecklingen av såväl tekniken som samhälle så visar sig en tydlig trend mot digitalisering av allt från
post och medelanden till betalningar och personuppgifter. Detta har öppnat upp för helt
nya problem när det gäller säkerhet och integritet av information som inte tidigare funnits. Utan denna utbredning av
digitalisering så hade våran utveckling troligen begränsas men med den nya tekniken kommer även nya problem
som måste lösas.\footfullcite{diffie2010privacy}

Ett av dessa problem är integritet och säkerhet. Något som tidigare kunde lösas genom att låsa in informationen på
en fysisk plats men som nu inte längre är möjligt. Den digitala världen har gjort det nästan
omöjligt att vara helt säker och strävan efter att behålla den enskilda individens integritet
är en av de största utmaningarna som vi står inför idag.\footcite{diffie2010privacy}

\subsection{Kryptografins uppkomst} % Kind of done but need to ask jimmy of what he thinks and maybe someone more
Kryptografins historia kan man nästan säga börjar vid den
tidigaste formen av skrift, vilket grundar sig i de faktum att
de flesta inte kunde läsa. Detta är ju såklart något som förändrats
på senare tid och i takt med de så har även kryptografin utvecklats.
Exempel på utvecklingen går att se så tidigt som 1900 f.Kr då vissa egyptiska
skribenter använde sig utav hieroglyfer på ett avvikande sätt, vilket
troligen då gjordes i syfte att dölja informationen från dom som inte
visste vad det skulle betyda.\footcite{kryptografi-historia-1}

Den tidiga kryptografin är även något som kan observeras hos romarna där
man använde \gls{caesar} och hos grekerna. Där grekernas metod byggde på
att man virade en pappersbit runt någon form av ett cylinderformat objekt
och sedan skrev medellandet på pappersbiten. När pappersbiten sedan togs av så är texten % This section might require some modification
oläslig och mottagaren behövde vira upp pappersbiten på ett cylinderformat objekt
med samma diameter för att läsa det.\footcite{kryptografi-historia-1}

\subsection{Kryptografins utveckling} % May need more but not sure
Utvecklingen av kryptografin som en vetenskap och teknik såg dock inga större framsteg
ända till medeltiden. När utvecklingen ändå började ta fart igen så använde bland annat
nästan alla Europeiska nationer någon form av kryptografi för att dölja medelande och hemlig kommunikation.
Under den här tiden utvecklades bland annat \gls{polyalphabetic-substitutionsskiffer} där ett av dom tidigaste skapades av
Leon Battista Alberti.\footcite{kryptografi-historia-1}

Där efter så forsattes \gls{polyalphabetic-substitutionsskiffer} att användas och utvecklas
under många år fram till 1900 då bland annat \gls{enigma} uppkom. \gls{enigma} var ett krypteringsverktyg som
bygger på \gls{substitutionsskiffer} precis som många skiffer tidigare men som tills skillnad från tidigare
använde sig av ett flertal nya metoder för att göra krypteringen säkrare.\footcite{kryptografi-historia-1}

\gls{enigma} kan man nästan se som ett av de första stegen i utvecklingen av den moderna kryptografin som
till stora delar bygger på våran teknologiska utveckling. Den nya tekniken öppnade nya portar, vilket bland annat gjorde det möjligt
för krypteringen att bli mer komplicerad och säkrare utan att påverkar användbarheten. Men utvecklingen visades sig även inom
dekrypteringen där ett tydligt exempel är hur en av de första fullt programmerbara datorerna Colossus skapades. Datorn hade i syfte
att användes i arbetet med att dekryptera medelande skickade av Tyskarna under andra världskriget och spelade på så sätt
en ganska viktigt roll i historien.\footfullcite{krypto}

Senare in på 1900-talet och tidigt 2000-tal så har kryptografin utvecklats ytterligare och idag finns otaliga
algoritmer och system som används för att kryptera medelanden. Där ibland bland annat algoritmer som \gls{aes} och \gls{des} men även
protokoll som \gls{http} och \gls{ssh}.\footcite{krypto}

\section{AES Uppkomst} % The rise of the AES standard and the Rijndael algorithm
\label{sec:aes-uppkomst}
Startskottet för uppkomsten av \acrshort{aes} gavs av \acrfull{nist} som 1997
utlyste en utmaning för att skapa en ny standard för kryptering för att ersätta
\acrshort{des} som då var den dominerande standarden.\footfullcite{nechvatal2001report} Utmaningen utlystes för att
\acrshort{des} säkerhet började bli allt mer ifrågasatt i takt med att datorerna blev mer kraftfulla, vilket då blev starten för sökandet efter
en ny mer framtidssäker standard.\footfullcite{burr2003selecting}

\acrshort{nist} utlyste sedan 1998 de 15 kandidaterna som valts ut. Där efter så fick
den kryptografiska forskargruppen runt om i världen möjligheten att undersöka och testa
de olika kandidaterna under processen. Efter ett flertal rundor av analysering och testande
där antalet kandidater sakta men säkert minskat så valdes tillslut 5 kandidater ut som
finalister. Dessa var Rijndael, RC6, Serpent, MARS och Twofish. Slutligen en tid senare så
valdes Rijndael ut som den nya standarden och en modifierad verison av Rijndael
blev då sedan den så kallade \acrfull{aes}.\footcite{nechvatal2001report}
