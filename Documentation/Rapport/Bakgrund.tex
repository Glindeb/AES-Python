\mychapter{Bakgrund}

\section{section 1}
Ifall man nu vänder sig mot de möjliga felkällorna som finns kan man exempelvis ta upp de faktum
att kaliumpermanganatlösningens (KMnO4) koncentration kan ha påverkats av ljuset i rummet.
Bland annat skulle detta kunna ske från de att lösningen hällts ut ur sina mörka flaskor och i byretten
som till skillnad från flaskorna är klart genomskinlig och stoppar inget ljus. Detta skulle i och med
detta kunna vara en möjlig systematisk felkälla som kan ligga till grunden till varför många av
värden blev alldeles för stora och till och med översteg 100\% vilket är ett helt orimligt resultat.
En liknande systematisk felkälla skulle även kunna vara att exaktheten av koncentrationen på
kaliumpermanganatlösningen (KMnO4) kan ha varierat mellan behållare, vilket då kan vara en av
anledningarna till de mönster så syns bland resultat värdena. Detta då vissa av resultaten på ett sätt
ser ut att gruppera sig i grupper där värden inte varierar hur mycket som helst mellan varandra.
Bland annat så kan man se detta väldigt tydligt bland de rödmarkerade värdena som är över 100\%
där dom mellan sig skiljer sig som mest med ca 20 procentenheter. Detta medans det minsta värdet
av de rödmarkerade skiljer sig med ca 20 procentenheter till de högsta grön markerade värdet.
Samtidigt som skillnaden bland de grön markerade värdena maximalt ligger på ca 20 procentenheter.
Ytterligare en felkälla skulle kunna vara avläsningen av volymen kaliumpermanganatlösning
(KMnO4) konsumerad vid titreringen. Avläsningen är en sak som bygger på en fysisk observation av
en person som får avgöra vad volymen blir utifrån skalan på byretten och vätskenivån. Detta
introducerar då en slumpmässig felkälla som påverkar resultatets säkerhet negativt. Något som även
ytterligare ökar påverkan från denna felkälla är det faktum att avläsningen ytterligare försvårades på
grund av kaliumpermanganatlösningens (KMnO4) mörka färg som gjorde de svårt att avläsa skalan
på byretten. Denna felkälla får även en ännu större betydelse ifall man även var tvungen att fylla på
byretten en extra gång under titrerings processen, detta då man exempelvis får två värden som de kan
uppkomma ett visst fel på samtidigt som de även introducerar ett tredje värde med en viss osäkerhet
då man möjligen inte riktigt lyckas fylla byretten till ett exakt sträck på skalan.
Ännu en felkälla skulle även kunna vara bestämmelsen av när man nått ekvivalens punkten då man
enligt metoden når ekvivalens punkten när den titrerade lösningen blir rosalila i minst 30 sek efter en
droppe. Detta innebär att även ifall man använder ett tidtagarur för att ta tiden på färgomslaget så
introduceras ytterligare en osäkerhet. Bland annat beror de på att man kan uppfatta vad som är rätt
färg för när ekvivalens punkten är nådd olika från person till persons, vilket då skapar en viss
subjektivitet då lösningen inte helt i vissa fall blir en helt exakt klar färg av rosalila direkt. I och med
detta så skulle de kunna ha en påverkan på resultatet och på så vis även de slutliga
resultatets säkerhet. Denna felkälla får även en ännu större betydelse ifall man även var tvungen att fylla på
byretten en extra gång under titrerings processen, detta då man exempelvis får två värden som de kan
uppkomma ett visst fel på samtidigt som de även introducerar ett tredje värde med en viss osäkerhet
då man möjligen inte riktigt lyckas fylla byretten till ett exakt sträck på skalan.
Ännu en felkälla skulle även kunna vara bestämmelsen av när man nått ekvivalens punkten då man
enligt metoden når ekvivalens punkten när den titrerade lösningen blir rosalila i minst 30 sek efter en
droppe. Detta innebär att även ifall man använder ett tidtagarur för att ta tiden på färgomslaget så
introduceras ytterligare en osäkerhet. Bland annat beror de på att man kan uppfatta vad som är rätt
färg för när ekvivalens punkten är nådd olika från person till persons, vilket då skapar en viss
subjektivitet då lösningen inte helt i vissa fall blir en helt exakt klar färg av rosalila direkt. I och med
detta så skulle de kunna ha en påverkan på resultatet och på så vis även de slutliga
resultatets säkerhet.

\subsection{section 2}
abstract
\subsubsection{section 3}
abstract