\mychapter{Diskussion} % Diskussion av resultat, presentera och tolka resultatet samt för ett nyanserat resonemang om vad resultatet betyder. Diskutera även eventuella brister i experimentet och hur dessa kan förbättras.
\label{chap:discussion}

% Tolka resultatet från nyckel längds undersökningen (To be rewived)
Utifrån resultatet från undersökningen går det att se hur tiden det tar att kryptera en fil på 1MB ökar vid användandet av längre nycklar så som 192-\gls{bit} och 256-\gls{bit} nycklar
i förhållande till en 128-\gls{bit} nyckel. Något som bland annat går att se hos procentuella tidsskillnaden mellan det olika nyckel längderna. Där det går att observera en
tidsskillnad på 18,8\% mellan 128-\gls{bit} \& 192-\gls{bit} nyckeln och 27,8\% mellan 128-\gls{bit} \& 256-\gls{bit} nyckeln.

Anledningen till denna tidökning är något som exempelvis kan bero på antalet rundor som genomförs för varje 16-\gls{byte} block som krypteras, vilket är den huvudsakliga
skillnaden mellan det olika nyckel längderna. Eftersom för en 128-\gls{bit} nyckel så genomförs 10 rundor för varje 16-\gls{byte} block som krypteras, medan för en 192-\gls{bit}
nyckel så genomförs 12 rundor och för en 256-\gls{bit} nyckel så genomförs 14 rundor. Något som då höjer antalet operationer som genomförs för varje 16-\gls{byte} block, vilket
i sin tur då troligen höjer den totala krypterings tiden.

% Tolka resultatet från körläges undersökningen (To be rewived)
Resultatet visar även att skillnaden i krypteringstid mellan olika körlägen är väldigt liten. Mellan \acrshort{ecb} \& \acrshort{cbc} var skillnaden 0,3\% och mellan \acrshort{ecb}
\& \acrshort{ofb} var skillnaden 0,4\%. Något som inte riktigt var väntat då det för både \acrshort{cbc} och \acrshort{ofb}
genomförs en ytterligare operation mellan varje 16-\gls{byte} block som krypteras för att länka ihop de olika blocken. Men samtidigt så är denna extra operation en \gls{xor}-operation
som är relativt lätt och snabb för en dator att genomföra. Något som då innebär att trots att det genomförs en extra operation så är den ändå relativt snabb och därför inte
påverkar krypterings tiden särskilt mycket.

I resultatet av körläges testet kan det dock även observeras viss spridning i tiden som det tar att kryptera en fil mellan det olika omgångarna. Detta bland annat vid jämförelse av
max och min tiderna för det olika körlägena. Exempelvis maxvärdet för \acrshort{ecb} som är 21,0857602999968 s, vilket är högre än
maxvärdet för \acrshort{cbc} som är 20,9346826999972 s och \acrshort{ofb} som är 20,8816630000001 s. En skillnad som då påvisar att det finns en viss osäkerhet i
resultatet, något som även ytterligare bekräftas av min värdena. I och med detta går det inte med säkerhet att konstatera något utifrån resultatet av körläges testet
när det gäller vilken betydelse det har för krypterings tiden.

% Tolka resultatet från krypteringsmetod undersökningen (To be rewived)
När det gäller hur säkerheten påverkas av det olika körlägena så går det ganska tydligt att se hur \acrshort{ecb} är det mest osäkra körläget för större informationsmängder.
Detta eftersom det i resultatet kan observers hur det trots kryptering fortfarande går att se spår av den ursprungliga bilden i informationen som krypterats. Något som inte
går att göra när bilden istället krypteras med hjälp av \acrshort{cbc} eller \acrshort{ofb}.

% Utvärdera metoden (nyanserat resonemang) (To be rewied)
Själva metoden som användes för att genomföra undersökningen bär med sig både för och nackdelar. Bland annat så medför metoden en ökad förståelse för hur \acrfull{aes} fungerar
på en låg nivå tack vare det faktum att implementeringen av algoritmen gjordes specifikt för undersökningen. En förståelse som gör det lättare att formulera resonemang och
dra slutsatser om hur \acrshort{aes} fungerar. Vilket är något som skulle gå förlorat ifall vid användandet av en befintlig implementation av \acrshort{aes}. Men att implementera
\acrshort{aes} på egenhand innebär också att det finns en risk för att implementeringen av algoritmen inte är helt korrekt och är även en tidskrävande process som kräver
mycket arbete. Något som kan ses som en nackdel som skulle kunna undvikas ifall en befintlig implementation av \acrshort{aes} använts.

En fördel med metoden när det gäller just nyckellängds testet är att säkerheten hos resultatet stärks genom att varje nyckel testas flera gånger, vilket då medför att
potentiell slumpmässiga felkällor som kan påverka resultatet minskas. Men samtidigt så innebär detta att tiden som krävs för att genomföra undersökningen ökar. Något som
då påverkar hur stora filer som går att testa för att kunna genomföra undersökningen inom en rimlig tidsrymd.

Ytterligare fördelar med metoden är att den till stor del är automatiserad, vilket då minskar den mänskliga faktorns påverkan. Samtidigt som det gör det lättare att
repetera undersökningen fler gånger med liten variabilitet, vilket då resulterar i ett mer tillförlitligt resultat. Sedan så är
en annan fördel att det lätt går att jämföra och urskilja skillnader i säkerheten för stora datamängder mellan olika körlägen tack vare att en bild användes som testdata.
Vilket då ger en tydlig visuell indikation på hur säkerheten påverkas av det olika körlägena.

\section{Felkällor} % (To be rewived)
\label{sec:errors}
När det gäller felkällor så finns det bland annat som nämnts tidigare en risk för att implementeringen av \acrshort{aes} inte är helt korrekt. Vilket
då skulle kunna påverka resultatets tillförlitlighet. Bland annat genom att introducera tidsskillnader mellan exempelvis olika körlägen eller nyckel längder som inte skulle finnas i en korrekt implementering. Detta innebär
då att det finns en risk att fel i implementeringen av \acrshort{aes} kan påverka resultatet, vilket därmed tillför en viss osäkerhet till resultatet. % (To be rewived)

En annan felkälla skulle även kunna vara själva resultat hanteringen. Något som för denna undersökning gjordes manuellt efter att undersökningen var genomförd. Detta innebär då att
det finns en möjlighet för fel som beror på den mänsklig faktorn som exempelvis felaktig avläsning eller felaktiga beräkningar vid sammanställning av resultatet. Något som då ytterligare
påverkar resultatets tillförlitlighet negativt. % (To be rewived)

Sedan skulle en annan felkälla även kunna vara en variation i \gls{cpuh} mellan omgångarna, vilket då resulterar i att olika omgångar av undersökningen hinner olika många instruktioner
per sekund. En faktor som då innebär att det kan ta olika lång tid för samma operationer att utföras mellan omgångarna i undersökningen. Detta leder då till en viss variation i resultatet,
vilket då även till för en viss osäkerhet. % (To be rewived)

Slutligen skulle även ytterligare en felkälla kunna vara \nameref{sec:aes-key-expansion}en som användes för att generera nycklarna för varje runda. Detta eftersom detta steg behöver genomföra fler operationer för längre nyckel längder.
Något som då innebär att det tar längre tid för större nycklar att utökas i förhållande till kortare nycklar. Detta är då en systematisk felkälla som innebär att det alltid kommer
finnas ett visst tidstillägg för större nycklar i förhållande till kortare nycklar. Ett faktum som behöver tas i beaktande vid jämförelser mellan olika nyckel längder. % (To be rewived)

\section{Förbättringar} % (To be rewived)
\label{sec:improvements}
När det gäller möjliga förbättringar av undersökningen så skulle en förbättring kunna vara att öka antalet gånger som varje nyckel och körläge testas. Detta skulle då
minska risken för att resultatet blir påverkat av slumpmässiga felkällor så som exempelvis variationer i klockhastigheten. Något som då skulle göra resultatet mer tillförlitligt. % (To be rewived)

En annan förbättring skulle även kunna vara att genomföra \nameref{sec:aes-key-expansion}en innan själva tidtagningen av krypteringen. Något som då skulle kunna eliminera den systematiska
felkälla som nämnts tidigare när det gäller \nameref{sec:aes-key-expansion}en. Detta innebär då att jämförelser av tiden det tar att kryptera något mellan olika nyckel längder skulle spegla
den faktiska skillnaden bättre och därmed ge ett mer tillförlitligt resultat. % (To be rewived)

Det är dock värt att notera att denna skillnad som kan uppstå mellan olika nyckel längder för \nameref{sec:aes-key-expansion}en inte är särskilt stor i förhållande till resten av krypteringsprocessen
i takt med att data mängden ökar. Detta innebär då att felet som uppstår från att genomföra \nameref{sec:aes-key-expansion}en medans tidtagningen pågår även skulle kunna minimeras genom att
öka data mängden. Något som då skulle göra att skillnaden blir försumbar i förhållande till resten av krypteringsprocessen.% (To be rewived)

Ytterligare en förbättring kan vara att på något sätt begränsa \gls{cpuh} till en viss fast klockhastighet, vilket då skulle ge varje omgång av undersökningen samma förutsättningar
att hinna med exakt samma antal operationer per sekund. Något som då skulle innebära ett mer tillförlitligt resultat. Utöver detta skulle även den mänskliga faktorns påverkan på resultatet kunna minskas
genom att automatisera resultat hanteringen och beräkningarna av exempelvis medelvärde. Något som då även det skulle göra resultatet mer tillförlitligt. % (To be rewived)

En annan förbättring skulle även kunna vara att kontrollera implementeringen av \acrshort{aes} ytterligare exempelvis genom att låta någon utomstående individ insatt i ämnet och
med en bra förståelse utav programmeringsspråket \gls{python} gå igenom koden rad för rad. Något som då skulle kunna minimerar risken för felaktigheter i implementeringen,
vilket i sin tur då skulle leda till att resultatet blir mer tillförlitligt. % (To be rewived)

\section{Slutsats} % (To be rewived)
\label{sec:conclusion}
Ska nu frågeställningen “Hur påverkas tiden det tar att kryptera något mellan det olika nyckel längderna 128-bit, 192-bit och 256-bit nyckel?” besvaras så visar resultatet från
undersökningen tydligt hur tiden det tar att kryptera något ökar ganska mycket i takt med att nyckeln blir längre. En tids ökning på 18,8\% mellan 128-bit och 192-bit nyckel och en tids ökning på 27,8\%
mellan 192-bit och 256-bit nyckel. Ett resultat som trots vissa felkällor och osäkerheter ändå är tillräckligt tillförlitligt för att kunna dra slutsatsen att det tar längre tid.

Sedan när frågeställningen “Hur förändras krypterings tiden mellan de olika körlägena \acrshort{ecb}, \acrshort{cbc} \& \acrshort{ofb}?” ska besvaras så är resultatet inte
riktigt lika tydligt som när det gäller nyckel längderna. Resultatet visar på en tidsökning på 0,3\% mellan \acrshort{ecb} och \acrshort{cbc} och en tidsökning på 0,1\% mellan \acrshort{cbc} och
\acrshort{ofb}. Att dra några konkreta slutsatser från detta resultat är dock inte möjligt utifrån denna undersökning då det är svårt att säga om skillnaden mellan \acrshort{ecb} och \acrshort{cbc} samt
\acrshort{cbc} och \acrshort{ofb} är på grund av en faktisk tidsskillnad eller om det är ett resultat av felkällor och osäkerheter.

Ett faktum som ganska tydligt framgår i hur skillnaden mellan de högsta uppmätta värdena och det lägst uppmätta värdena för varje körläge. Där det går att se hur
exempelvis det högsta värdet för \acrshort{ecb} är högre än både det högsta värdet för \acrshort{cbc} och det högsta värdet för \acrshort{ofb}, samtidigt som det lägsta värdet för
\acrshort{ofb} är lägre än både det lägsta värdet för \acrshort{ecb} och det lägsta värdet för \acrshort{cbc}.

Slutligen kan frågeställningen “Hur påverkas skiffertexten av det olika körlägena \acrshort{ecb}, \acrshort{cbc} \& \acrshort{ofb} samt vilken betydelse får det ur ett säkerhetsperspektiv?”
besvaras med hjälp av resultatet från krypteringstestet där det tydligt gå att se hur skiffertexten förändras beroende på vilket körläge som används. Något som visar sig hos skiffertexten
från \acrshort{ecb} krypteringen där det tydligt går att se spår av den ursprungliga bilden samtidigt som detta inte går att se i skiffertexten från \acrshort{cbc} och \acrshort{ofb}. Utifrån detta går det då att dra slutsatsen
att körläget \acrshort{ecb} är sämre ur ett säkerhetsperspektiv för större datamängder än \acrshort{cbc} och \acrshort{ofb} eftersom det trots kryptering fortfarande går att se spår av den ursprungliga
informationen i skiffertexten.

Sammanfattningsvis går det utifrån denna undersökning då att dra slutsatsen att det tar längre tid att kryptera information vid användande av en längre så som 256-bit nyckel jämfört med en kortare så som 128-bit nyckel. Samtidigt
som det möjligen skulle kunna finnas en liten tidsskillnad mellan \acrshort{ecb}, \acrshort{cbc} och \acrshort{ofb}. Samt att \acrshort{ecb} är sämre ur ett säkerhetsperspektiv för att kryptera större informationsmängder än
\acrshort{cbc} och \acrshort{ofb}.

