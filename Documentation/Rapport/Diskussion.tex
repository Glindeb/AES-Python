\mychapter{Diskussion} % Diskussion av resultat, presentera och tolka resultatet samt för ett nyanserat resonemang om vad resultatet betyder. Diskutera även eventuella brister i experimentet och hur dessa kan förbättras.
\label{chap:discussion}

% Tolka resultatet från nyckellängds undersökningen
Utifrån resultatet från undersökningen kan man då se hur tiden det tar att kryptera en fil på 1MB ökar när man använder längre nycklar så som 192-\gls{bit} 256-\gls{bit} nycklar
i förhållande till en 128-\gls{bit} nyckel. Något som bland annat visar sig när man tittar på den procentuella tidskillnaden mellan de olika nyckellängderna. Där man kan se en
tids skillnad på 18,8\% mellan 128-\gls{bit} \& 192-\gls{bit} nyckeln och 27,8\% mellan 128-\gls{bit} \& 256-\gls{bit} nyckeln.
Anledningen till detta är något som högst troligen beror på att antalet krypteringsrundor som genomförs för det olika nyckellängderna ökar när nyckeln blir längre.
Något som då höjer antalet operationer som genomförs för varje 16-\gls{byte} block som krypteras, vilket i sin tur höjer den totala krypterings tiden.

% Tolka resultatet från körläges undersökningen
Resultatet visar även att skillnaden i krypteringstid mellan olika körlägen är väldigt liten. Mellan \acrshort{ecb} \& \acrshort{cbc} var skillnaden 0,3\% och mellan \acrshort{ecb}
\& \acrshort{ofb} var skillnaden 0,4\%. Något som inte riktigt var väntat då det för både \acrshort{cbc} och \acrshort{ofb}
genomförs en ytterligare operation mellan varje 16-\gls{byte} block som krypteras för att länka ihop de olika blocken. Men samtidigt så skulle anledning till att skillnaderna är
så små kunna vara att \gls{xor}-operationen som används för att länka ihop blocken är en operation som är relativt lätt och snabb för en dator att genomföra, samtidigt som
operationen även bara behöver genomföras en gång för varje 16-\gls{byte} block som krypteras.

I resultatet av körläges testet kan man dock även se en viss spridning i tiden som det tar att kryptera en fil mellan det olika omgångarna.
Något som visar på en viss osäkerhet då \acrshort{ecb} har en högsta tid som är högre än både den högsta tiden för \acrshort{cbc} och \acrshort{ofb}. Vilket då skulle kunna antyda
att de skillnader som mätts upp mellan medelvärdena kan vara något som är inom felmarginalen för mätningen. Detta innebär då att det inte går att säga med säkerhet att
det finns någon större skillnad i krypterings tiden mellan det olika körlägena utifrån denna undersökning.

% Tolka resultatet från krypteringsmetod undersökningen
Tittar man sedan istället på hur säkerheten påverkas av det olika körlägena så kan man ganska tydligt se att \acrshort{ecb} är det mest osäkra körläget för större data mängder.
Detta då man som visas i resultatet kan se hur trots kryptering man fortfarande kan se spår av den ursprungliga bilden i informationen som krypterats. Något som inte
går att göra när bilden istället krypteras med hjälp av \acrshort{cbc} eller \acrshort{ofb}.

% Utvärdera metoden (nyanserat resonemang)



\section{Felkällor} % Variabilitet i klockhastighet, Ogämnheter i implementeringen som resulterar i missvisande tider, osäkerhet host mätmetoden, fel i implementeringen,


\section{Förbättringar} % Öka antalet reppetitioner för att minskar risken för fel, använda andra metoder för att mäta tiden, testa olika nycklar för att se hur de påverkar,


\section{Slutsats}

