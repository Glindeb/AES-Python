\mychapter{Diskussion \& Slutord} % Diskussion av resultat, presentera och tolka resultatet samt för ett nyanserat resonemang om vad resultatet betyder. Diskutera även eventuella brister i experimentet och hur dessa kan förbättras.
\label{chap:discussion}

Utifrån resultatet från undersökningen kan man då se hur tiden det tar att kryptera en fil på 1MB ökar när man använder längre nycklar så som 192-\gls{bit} 256-\gls{bit} nycklar.
Samtidigt visar det sig även att skillnaden i krypteringstid mellan olika körlägen är väldigt liten. Något som tydligt visas det procentuella tids skillnaderna där
det som högst skiljde sig med 0,4\% mellan \acrshort{ecb} \& \acrshort{ofb}. Medans skillnaden mellan \acrshort{cbc} \& \acrshort{ofb} var 0,1\%.

Tittar man sedan istället på hur säkerheten påverkas av det olika körlägena så kan man ganska tydligt se att \acrshort{ecb} är det mest osäkra körläget för större data mängder.
Detta då man som visas i resultatet kan se hur trots kryptering man fortfarande kan se spår av den ursprungliga bilden i informationen som krypterats. Något som inte
går att göra när bilden istället krypteras med hjälp av \acrshort{cbc} eller \acrshort{ofb}.


\section{Felkällor} % Variabilitet i klockhastighet, Ogämnheter i implementeringen som resutlerar i missvisande tider, osäkerhet host mätmetoden, fel i implementeringen, 


\section{Förbättringar} % Öka antalet reppetitioner för att minskar risken för fel, använda andra metoder för att mäta tiden, testa olika nycklar för att se hur de påverkar, 


\section{Slutsats}


\section{Slutord}
