\mychapter{Metod \& Genomförande}
Metoden för denna undersökning bygger på en implementering av \acrshort{aes} i programmeringsspråket
\gls{python}. Detta tillsammans med ett antal konstruerade tester även dom implementerade i
\gls{python} är vad som använts för själva undersökningen av \acrshort{aes}. Koden
är skriven med hjälp av programmet \gls{vscode} och är byggd huvudsakligen för \gls{python} 3.10 men
på grund av att \gls{python} 3.11 släpptes innan undersökningen genomfördes så är de istället \gls{python} 3.11
som användes under undersöknings genomförandet.\footfullcite{python311}

\section{Implementering} % Describe the implementation of the AES algorithm in python
Implementeringen av \acrshort{aes} är uppdelad i ett antal funktioner till stor del är baserat på
hur strukturen och uppdelningen av \acrshort{aes} som beskrivet i \citetitle{daemen1999aes}\footcite{daemen1999aes}.
Några av det Huvudsakliga funktionerna av algoritmen är som följande:
\begin{itemize}
    \item \texttt{\nameref{sec:aes-subbytes}}
    \item \texttt{\nameref{sec:aes-shiftrows}}
    \item \texttt{\nameref{sec:aes-mixcolumns}}
\end{itemize}

Dessa funktioner används i varje runda och är grunden för hur algorithmen fungerar. Utöver detta finns
äne \nameref{sec:aes-addroundkey} som används för att lägga till nyckeln i varje runda. AddRoundKey steget är dock inte
en funktion eftersom det enbart består ut av en \gls{xor}-operation. Utöver detta finns det även
en funktion som används för att expandera nyckeln som beskrivs i \nameref{sec:aes-key-expansion} samt
individuella funktioner för kryptering och dekryptering till de olika körlägen \nameref{sec:ecb}, \nameref{sec:cbc} och
\nameref{sec:ofb}.

Implementeringen använder sig av \gls{python} biblioteket NumPy för att hantera matriser och genomföra
matematiska operationer på dessa. Hela implementeringen går att hitta i appendix \ref{app:python} "\nameref{app:python}".

\section{Test Uppsättning} % Describe the individual test setup for each test and it´s method
Test uppsättningen är uppdelad i tre delar där varje del är konstruerad för att generera ett
resultat i koppling till frågeställningarna för denna undersökning. Två av delarna, \nameref{sec:nyckellängd-test} och
\nameref{sec:körlages-test} är skrivna som ett \gls{python} script som kan köras för att då genererar
resultatet. När det gäller Krypteringstest så skiljer sig det från de två andra testerna eftersom det
inte är ett skript utan ett antal manuella steg som genomfördes för att generera resultatet istället.

\subsection{Nyckellängds Test}
\label{sec:nyckellängd-test}


\subsection{Körläges Test}
\label{sec:körlages-test}


\subsection{Krypterings Test} % explain how the encryption test is setup
\label{sec:krypterings-test}
Krypterigns testet består utav att man krypterar en bild i \gls{ppm} format med \acrshort{aes} i tre olika körlägen.
PPM formatet används då de tack vare sin enkla struktur kan genomgå en kryptering och fortfarande representeras som en bild
efteråt. Detta gör det då möjligt att visualisera resultatet av krypteringen. För att kryptera bilden används den implementerade
\acrshort{aes} algorithmen som körs i körlägena \nameref{sec:ecb}, \nameref{sec:cbc} och \nameref{sec:ofb} körlägen.

Före och efter krypteringen modifieras dock \gls{ppm} filen så att den går att öppna när den väl är krypterad. Detta görs genom att de fyra
första raderna i filen tas bort och sätts sedan tillbaka efter krypteringen. Detta gör att filen fortfarande kan öppnas som en bild
efter krypteringen.

\section{Genomförande} % Describe the test enviroment that was used as well as hwo the test was preformed
Först kördes Nyckellängdstestet och Körlägestestet genom att köra \gls{python} filen \nameref{app:analyze}. Där efter
genomfördes Krypterings testet. För krypteringstestet valdes först en lämplig bild för krypteringen ut. Sedan
konverterades bilden till \gls{ppm} format med hjälp av programmet \gls{gimp}.