\mychapter{Metod \& Genomförande}
Metoden för denna undersökning bygger på en implementering av \acrshort{aes} i programmeringsspråket
\gls{python}. Detta tillsammans med ett antal konstruerade tester även dom implementerade i
\gls{python} är vad som använts för själva undersökningen av \acrshort{aes}. Själva koden
är skriven med hjälp av programmet \gls{vscode} och är byggd huvudsakligen för \gls{python} 3.10 men
på grund av att \gls{python} 3.11 släpptes innan undersökningen genomfördes så är de \gls{python} 3.11
som användes under undersöknings genomförandet.\footfullcite{python311}

\section{Implementering}
Implementeringen av \acrshort{aes} är uppdelad i ett antal funktioner till stor del är baserat på
hur strukturen och uppdelningen av \acrshort{aes} som beskrivet i \citetitle{daemen1999aes}\footcite{daemen1999aes}.
De Huvudsakliga funktionerna är följande:
\begin{itemize}
    \item \texttt{\nameref{sec:aes-subbytes}}
    \item \texttt{\nameref{sec:aes-shiftrows}}
    \item \texttt{\nameref{sec:aes-mixcolumns}}
    \item \texttt{\nameref{sec:aes-addroundkey}}
    \item \texttt{\nameref{sec:aes-key-expansion}}
\end{itemize}

Dessa används då i själva algoritmens rundor samt utanför och är beskrivna och förklarade i respektive teori avsnitt.
Utöver detta kan

\section{Test Uppsättning}
Test uppsättningen är uppdelad i tre delar där varje del är konstruerad för att generera ett
resultat i koppling till frågeställningarna för denna undersökning.


\subsection{Nyckellängds Test}
\label{sec:nyckellängd-test}


\subsection{Körläges Test}
\label{sec:körlages-test}


\subsection{Krypterings Test}
\label{sec:krypterings-test}


\section{Genomförande}
Genomförandet av undersökningen gick som följande: