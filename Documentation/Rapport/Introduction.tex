\mychapter{Introduktion} %  Need to be worked on!!!

Dator, mobiltelefon och till och med bil är några uta av dom saker som idag alla är uppkopplade
till internet. Information om allt från position och uppdateringar till kockslag och väder skickas
från enhet till enhet runt om oss. Utav detta så är mycket vad vi kallar privat information som
vi helst inte vill att vem som helst ska kunna se. På grund av just denna anledning så används
något som kallas kryptering för att dölja informationen från nyfikna mellanhänder.

I mordentid så är det absolut vanligast att \acrfull{aes} används för att kryptera våran information.


\section{Syfte} % Mostly clear but might need some refinement
Syftet med denna undersökning är att undersöka krypterings algoritmen \acrshort{aes},
för att utveckla en förståelse för mer avancerad krypterings algoritmer.
Samt att bygga en uppfattning om hur man på olika sätt kan implementera
krypterings algoritmer och vad de får för betydelse för deras säkerhet och
hastighet.

\section{Frågeställningar} % Mostly clear but might need some refinement and reformulating
\begin{itemize}
    \setlength{\itemindent}{-1em}
    \item Hur påverkas tiden de tar att kryptera något mellan de olika nyckel längderna 128-bit,
          192-bit och 256-bit nyckel?

    \item Hur påverkas algoritmen av de olika körlägen och vilken betydelse får de för den resultatet?

    \item Hur förändras tiden de tar att kryptera något beroende på ifall algoritmen körs i
          \acrshort{ecb}, \acrshort{cbc} eller \acrshort{ofb} samt vilken betydelse ur ett
          användnings perspektiv de får?
\end{itemize}

\section{Avgränsning} % Mostly clear but might need some refinement and reformulating

Denna rapport är inte en komplett utvärdering av \acrshort{aes} och dess användning. Utan fokuserar
enbart på hur nyckellängd och körläge påverkar krypteringstiden. Detta samt hur den resulterande
cipher texten påverkas av vissa körlägen och hur detta då i sin tur kan påverka säkerheten. \par

Denna analys av algoritmens säkerhet är alltså inte en komplett säkerhets utvärdering och tar inte
hänsyn till faktorer så som möjliga attacker där ibland exempelvis PLACEHOLDER \& PLACEHOLDER
attacker. Undersökningen är även begränsad till en mjukvaruimplementering och tar inte hänsyn
till möjliga skillnader som kan uppstå när algoritmen implementeras på en hårdvarunivå.