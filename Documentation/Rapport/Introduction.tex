\mychapter{Inledning} % Mostly clear but might need some refinement
Kryptering, en bärande grundsten i dagens digitaliserade samhälle. De är väggen mellan oss och resten
av världen, ett lås runt våra liv. Kryptering bygger på ett simpelt koncept, att dölja informationen
från all förutom den menade mottagaren. Ett koncept som exempelvis fanns redan för 2000 år sedan när Julius
Caesar använde de vi idag kallar \gls{caesar} för att skicka hemliga meddelanden.\footfullcite{luciano1987cryptology}
Sedan dess har kryptografi självklart utvecklats enormt och vi har gått från de på ett sätt simpla men
även eleganta \gls{caesar} som användes då till moderna algoritmer så som \acrlong{aes} och \acrlong{des}.
Dessa algoritmer har samma syfte som \gls{caesar} men har utvecklats under en tid där datorer står som
de dominerande informationshanteringsverktyget, vilket även är vad som används i denna rapport för att
undersöka just en av dessa algoritmer.

\section{Syfte} % Mostly clear but might need some refinement
Syftet med denna undersökning är att undersöka krypterings algoritmen \acrshort{aes},
för att utveckla en förståelse för mer avancerade krypterings algoritmer.
Samt att bygga en uppfattning om hur man på olika sätt kan implementera
krypterings algoritmer och vad de får för betydelse för deras säkerhet och
hastighet.

\section{Frågeställningar} % Mostly clear but might need some refinement and reformulating
\begin{itemize}
    \setlength{\itemindent}{-1em}
    \item Hur påverkas tiden de tar att kryptera något mellan de olika nyckel längderna 128-bit,
          192-bit och 256-bit nyckel?

    \item Hur påverkas skiffertexten av de olika körlägen och vilken betydelse får de för den resultatet?

    \item Hur förändras tiden det tar att kryptera något beroende på ifall algoritmen körs i
          \acrshort{ecb}, \acrshort{cbc} eller \acrshort{ofb} samt vilken betydelse det får ur ett
          tillämpningsperspektiv?
\end{itemize}

\section{Avgränsning} % Mostly clear but might need some refinement and reformulating

Denna rapport är en avgränsad utvärdering av \acrshort{aes} och dess användning som fokuserar
 på hur nyckellängd och körläge påverkar krypteringstiden. Detta samt hur den resulterande
skiffer texten påverkas av vissa körlägen och hur detta i sin tur kan påverka säkerheten. \par

Denna analys av algoritmens säkerhet utelämnar faktorer så som möjliga attacker där ibland exempelvis Brute-Force\footfullcite{kumar2011investigations}
\& Side-Channel\footfullcite{10.1007/978-3-642-04138-9_8} attacker. Undersökningen är även
begränsad till en mjukvaruimplementering och tar inte hänsyn till möjliga skillnader som kan uppstå
när algoritmen implementeras på en hårdvarunivå.