\mychapter{Inledning} % Mostly clear but might need some refinement
Kryptering, en bärande grundsten i dagens digitaliserade samhälle. Det är väggen mellan oss och resten
av världen, ett lås runt våra liv. Kryptering bygger på ett simpelt koncept, att dölja informationen
från all förutom den menade mottagaren. Ett koncept som har funnits med oss under stora delar av
människans historia och som än idag är en viktig del av vårt samhälle.\footfullcite{luciano1987cryptology}

Att dölja information har gått från de enkla metoder som användes redan för 2000 år sedan för att förmedla hemlig information med hjälp av
exempelvis \gls{caesar} till det komplexa och invecklade algoritmer som idag används för att skydda nästan all information som lagras och skickas
över internet.\footcite{luciano1987cryptology} Dessa komplexa algoritmer har utvecklats under en tid där datorer står som det dominerande informationshanteringsverktyget och
därför är det även en dator som används som huvudverktyg i denna rapport för att undersöka en av det vanligaste krypteringsalgoritmerna i dagens samhälle,
The \acrlong{aes}.

\section{Syfte} % Mostly clear but might need some refinement
Syftet med undersökning är att undersöka krypterings algoritmen \acrshort{aes},
för att utveckla en fördjupad förståelse för mer avancerade krypterings algoritmer.
Samt bygga en uppfattning om hur det på olika sätt går att implementera krypterings algoritmer
och vilken påverkan detta då får på deras prestanda och säkerhet.

\section{Frågeställningar} % Mostly clear but might need some refinement and reformulating
\begin{itemize}
    \setlength{\itemindent}{-1em}
    \item Hur påverkas tiden det tar att kryptera något mellan det olika nyckel längderna 128-bit,
          192-bit och 256-bit nyckel?

    \item Hur förändras krypterings tiden mellan de olika körlägena \acrshort{ecb}, \acrshort{cbc} \& \acrshort{ofb}?

    \item Hur påverkas skiffertexten av det olika körlägena \acrshort{ecb}, \acrshort{cbc} \& \acrshort{ofb}
          samt vilken betydelse får det ur ett säkerhetsperspektiv?
\end{itemize}

\section{Avgränsning} % Mostly clear but might need some refinement and reformulating

Denna undersökning är avgränsad till att endast undersöka \acrshort{aes} och dess
användning med fokus på hur nyckellängd och körläge påverkar krypteringstiden. Detta samt hur
den resulterande skiffer texten påverkas av det olika körlägena \acrshort{ecb}, \acrshort{cbc} \& \acrshort{ofb}
ur ett säkerhetsperspektiv.

Denna analys av algoritmens säkerhet tar inte hänsyn till faktorer så som möjliga attacker där
ibland exempelvis Brute-Force\footfullcite{kumar2011investigations}
\& Side-Channel\footfullcite{10.1007/978-3-642-04138-9_8} attacker. Undersökningen är även
begränsad till att endast utföras på en mjukvaruimplementering av \acrshort{aes}.