\mychapter{Inledning} %  Need to be worked on!!!

I takt med att vårat samhälle i allt större utsträckning digitaliserats så har behovet utav att kunna
hålla information privat också ökat. För att lösa detta så använder vi något kallat kryptering i syfte
att dölja och skydda vår information. Kryptering är något som idag används nästan överallt i form av olika
standarder så som \acrshort{aes} och \acrshort{des}. Vad vi en gör så påverkas vi på något sätt av den när
den skyddar våran information. På grund av detta så kan man förstå hur vikten av en grundläggande
förståelse kan vara något viktigt

\section{Syfte} % Mostly clear but might need some refinement
Syftet med denna undersökning är att undersöka krypterings algoritmen \acrshort{aes},
för att utveckla en förståelse för mer avancerad krypterings algoritmer.
Samt att bygga en uppfattning om hur man på olika sätt kan implementera
krypterings algoritmer och vad de får för betydelse för deras säkerhet och
hastighet.

\section{Frågeställningar} % Mostly clear but might need some refinement and reformulating
\begin{itemize}
    \setlength{\itemindent}{-1em}
    \item Hur påverkas tiden de tar att kryptera något mellan de olika nyckel längderna 128-bit,
          192-bit och 256-bit nyckel?

    \item Hur påverkas algoritmen av de olika körlägen och vilken betydelse får de för den resultatet?

    \item Hur förändras tiden de tar att kryptera något beroende på ifall algoritmen körs i
          \acrshort{ecb}, \acrshort{cbc} eller \acrshort{ofb} samt vilken betydelse ur ett
          användnings perspektiv de får?
\end{itemize}

\section{Avgränsning} % Mostly clear but might need some refinement and reformulating

Denna rapport är inte en komplett utvärdering av \acrshort{aes} och dess användning utan fokuserar
enbart på hur nyckellängd och körläge påverkar krypteringstiden. Detta samt hur den resulterande
chiffer texten påverkas av vissa körlägen och hur detta då i sin tur kan påverka säkerheten. \par

Denna analys av algoritmens säkerhet är alltså inte en komplett säkerhets utvärdering och tar inte
hänsyn till faktorer så som möjliga attacker där ibland exempelvis Brute-Force\footfullcite{kumar2011investigations}
\& Side-Channel\footfullcite{10.1007/978-3-642-04138-9_8} attacker. Undersökningen är även
begränsad till en mjukvaruimplementering och tar inte hänsyn till möjliga skillnader som kan uppstå
när algoritmen implementeras på en hårdvarunivå.